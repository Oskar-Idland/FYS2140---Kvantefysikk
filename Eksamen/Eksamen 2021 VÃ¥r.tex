\documentclass{article}
\usepackage{amsmath}
\usepackage[mathletters]{ucs}
\usepackage[utf8x]{inputenc}
\usepackage[margin=1.5in]{geometry}
\usepackage{enumerate}
\newtheorem{theorem}{Theorem}
\usepackage[dvipsnames]{xcolor}
\usepackage{pgfplots}
\setlength{\parindent}{0cm}
\usepackage{graphics}
\usepackage{graphicx} % Required for including images
\usepackage{subcaption}
\usepackage{bigintcalc}
\usepackage{pythonhighlight} %for pythonkode \begin{python}   \end{python}
\usepackage{appendix}
\usepackage{arydshln}
\usepackage{physics}
\usepackage{tikz-cd}
\usepackage{booktabs} 
\usepackage{adjustbox}
\usepackage{mdframed}
\usepackage{relsize}
\usepackage{physics}
\usepackage[thinc]{esdiff}
\usepackage{fixltx2e}
\usepackage{esint}  %for lukket-linje-integral
\usepackage{xfrac} %for sfrac
\usepackage{hyperref} %for linker, må ha med hypersetup
\usepackage[noabbrev, nameinlink]{cleveref} % to be loaded after hyperref
\usepackage{amssymb} %\mathbb{R} for reelle tall, \mathcal{B} for "matte"-font
\usepackage{listings} %for kode/lstlisting
\usepackage{verbatim}
\usepackage{graphicx,wrapfig,lipsum,caption} %for wrapping av bilder
\usepackage{mathtools} %for \abs{x}
\usepackage[norsk]{babel}
\definecolor{codegreen}{rgb}{0,0.6,0}
\definecolor{codegray}{rgb}{0.5,0.5,0.5}
\definecolor{codepurple}{rgb}{0.58,0,0.82}
\definecolor{backcolour}{rgb}{0.95,0.95,0.92}
\lstdefinestyle{mystyle}{
    backgroundcolor=\color{backcolour},   
    commentstyle=\color{codegreen},
    keywordstyle=\color{magenta},
    numberstyle=\tiny\color{codegray},
    stringstyle=\color{codepurple},
    basicstyle=\ttfamily\footnotesize,
    breakatwhitespace=false,         
    breaklines=true,                 
    captionpos=b,                    
    keepspaces=true,                 
    numbers=left,                    
    numbersep=5pt,                  
    showspaces=false,                
    showstringspaces=false,
    showtabs=false,                  
    tabsize=2
}

\lstset{style=mystyle}
\author{Oskar Idland}
\title{Eksamen 2021 Vår}
\date{}
\begin{document}
\maketitle
\newpage

\section*{Oppgave 1}
  \subsection*{a)}
  En egenfunskjon, er en funksjon som når opereres på av en operator, returnerer seg selv, multiplisert med en konstant. De er skarpt bestemt
  
  \subsection*{b)}
    Stasjonære tilstander er funksjoner hvis sannsynlighetsfordeling ikke endrer seg over tid. Alle fysiske observable er konstante over tid. 
    
\section*{Oppgave 2}
  \subsection*{a)}
    En normalisert bølgefunksjon betyr at integralet til sannsynlighetstettheten over intervallet partikkelen kan befinne seg i, er lik 1.
    
  \subsection*{b)}
    For å finne $\left<V\right>$ setter vi det inn i integralet. Dirac-deltaen gjør at vi kun får bidrag fra $x=0$, som gjør integralet lik $-α \frac{\sqrt{mα}}{ℏ}$. Videre vet vi at $\left<E\right> = \left<V\right> + \left<K\right>$. Vi finner først $\left<E\right>$ i grunntilstanden som vi vet er gitt ved $E = - \frac{mα^2}{2ℏ^2}$. 
    
    \[
    \left<V\right> = \left<E\right> - \left<K\right> = - \frac{mα^2}{2ℏ^2} + α \frac{\sqrt{mα}}{ℏ} 
    \]
    
    
    
    Videre løser vi får $\left<K\right>$
    \[
    \left<K\right> = ∫_{-∞}^{∞} \hat{K} \ \mathrm{d}x = \frac{1}{2m} ∫_{-∞}^{∞} (\hat{p}ψ)^{*} (\hat{p}ψ) \ \mathrm{d}x
    \]
    Vi vet at $\hat{p} = - iℏ \frac{∂}{∂x}$ og skriver om integralet for å ta hensyn til absolutt verdi. 
    \[
    \left<K\right> = \frac{1}{2m} \sqrt{\frac{mα}{ℏ}} \left(∫_{0}^{∞} e^{-mαx / ℏ^2} \ \mathrm{d}x ∫_{-∞}^{0} e^{mαx / h^2} \ \mathrm{d}x\right)
    \]
  
  \section*{c)}
  Vi finner forventningsverdien til $\left<x\right>$
  \[
  \left<x\right> =\frac{mα}{ℏ^2}  ∫_{-∞}^{∞} x e^{-2 \left|x\right| / ℏ^2} \ \mathrm{d}x 
  \]
  \[
  \frac{mα}{ℏ^2} \left( ∫_{0}^{∞} xe^{-2x / ℏ^2} \ \mathrm{d}x + ∫_{-∞}^{0} xe^{2x / ℏ^2} \ \mathrm{d}x \right)
  \]
  Vi gjør dette for $\left<x^2\right>$ også regner vi ut $\sqrt{\left<x^2\right> - \left<x\right>^2}$ for å finne standardavviket.
  
  Vi gjentar dette for $\left<p\right>$ og $\left<p^2\right>$. 
  

    
    
    


\end{document}