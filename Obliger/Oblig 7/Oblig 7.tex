\documentclass{article}
\usepackage{amsmath}
\usepackage[mathletters]{ucs}
\usepackage[utf8x]{inputenc}
\usepackage[margin=1.5in]{geometry}
\usepackage{enumerate}
\newtheorem{theorem}{Theorem}
\usepackage[dvipsnames]{xcolor}
\usepackage{pgfplots}
\pgfplotsset{compat=1.18}
\setlength{\parindent}{0cm}
\usepackage{graphics}
\usepackage{graphicx} % Required for including images
\usepackage{subcaption}
\usepackage{bigintcalc}
\usepackage{pythonhighlight} %for pythonkode \begin{python}   \end{python}
\usepackage{appendix}
\usepackage{arydshln}
\usepackage{physics}
\usepackage{tikz-cd}
\usepackage{booktabs} 
\usepackage{adjustbox}
\usepackage{mdframed}
\usepackage{relsize}
\usepackage{physics}
\usepackage[thinc]{esdiff}
\usepackage{fixltx2e}
\usepackage{esint}  %for lukket-linje-integral
\usepackage{xfrac} %for sfrac
\usepackage{hyperref} %for linker, må ha med hypersetup
\usepackage[noabbrev, nameinlink]{cleveref} % to be loaded after hyperref
\usepackage{amssymb} %\mathbb{R} for reelle tall, \mathcal{B} for "matte"-font
\usepackage{listings} %for kode/lstlisting
\usepackage{verbatim}
\usepackage{graphicx,wrapfig,lipsum,caption} %for wrapping av bilder
\usepackage{mathtools} %for \abs{x}
\usepackage[norsk]{babel}
\definecolor{codegreen}{rgb}{0,0.6,0}
\definecolor{codegray}{rgb}{0.5,0.5,0.5}
\definecolor{codepurple}{rgb}{0.58,0,0.82}
\definecolor{backcolour}{rgb}{0.95,0.95,0.92}
\lstdefinestyle{mystyle}{
    backgroundcolor=\color{backcolour},   
    commentstyle=\color{codegreen},
    keywordstyle=\color{magenta},
    numberstyle=\tiny\color{codegray},
    stringstyle=\color{codepurple},
    basicstyle=\ttfamily\footnotesize,
    breakatwhitespace=false,         
    breaklines=true,                 
    captionpos=b,                    
    keepspaces=true,                 
    numbers=left,                    
    numbersep=5pt,                  
    showspaces=false,                
    showstringspaces=false,
    showtabs=false,                  
    tabsize=2
}

\lstset{style=mystyle}
\author{Oskar Idland}
\title{Oblig 7}
\date{}
\begin{document}
\maketitle
\newpage

\section*{\underline{A Diskusjonsoppgaver}}
\subsection*{Oppgave 1}
\subsubsection*{a)}
Vi bruker utrykket for bølgefunksjonen til en fri partikkel som beskrevet i likning 2.95 i Griffiths.
\[
ψ(x) = Ψ(x,0) = Ae^{i\left(kx - \frac{ℏk^2}{2m} ⋅ 0\right)} = Ae^{ikx}
\]

\subsubsection*{b)}
For å utvide dette til den tidsavhengige bølgefunksjonen gjør vi følgende: 
\[
Ψ(x,t) = Ae^{i\left(kx - \frac{ℏk^2}{2m}t\right)}
\]

\subsubsection*{c)}
\[
∫_{-∞}^{∞} \left|Ψ(x,t)\right|^2 \ \mathrm{d}x = ∫_{-∞}^{∞} \left|A\right|^2 \ \mathrm{d}x = \text{Ikke definert}
\]

\subsubsection*{d)}
Er ikke normaliserbar og kan derfor ikke gi noe fysisk tolkning. 

\subsubsection*{e)}
Kvantefysikken løser dette ved å dele opp funksjonen til et multiplum at to funksjoner, $ψ$ som beskriver partikkelens posisjon og $ϕ$ som beskriver partikkelens bevegelsesmengde.

\subsection*{Oppgave 2}
\subsubsection*{a)}
Svaret er \underline{alternativ B: Bredden avtar}

\subsubsection*{b)}
$Ψ(x,0)$ forteller oss alt om partikkelens tilstand, men den beskriver ingen fysisk størrelse. Tar vi kvadratet av dets absolutt verdi får vi sannsynlighets fordelingen for å finne partikkelen i et område $x$ til et annet område $x + Δx$ ved tiden $t=0$. Derfor relaterer $Ψ$ til posisjonen til partikkelen.

\subsubsection*{c)}
Vi vet fra likning 2.95 i Griffiths at $k = ± \frac{\sqrt{2mE}}{ℏ}$. Ettersom dette er en fri partikkel vet vi at potensialet $V$ er null, som gjør at energien bare er avhengig av kinetisk energi $K$. Den kinetiske energien er gitt av bevegelsesmengden $p$ til partikkelen. 

\subsubsection*{d)}
Forholdet mellom spredningen til funksjonen $f$ som relaterer til partikkelens posisjon og spredningen til funksjonen $g$ som relaterer til partikkelens bevegelsesmengde er invers proporsjonalt som forventet fra Heisenberg's usikkerhetsprinsipp. Det er fordi standardavviket til en størrelse er proporsjonalt med spredningen til den tilhørende funksjonen.

\subsection*{Oppgave 3}
\subsubsection*{a)}
\begin{align*}
    Ψ(x,0) = Ae^{-a\left|x\right|}\\
    ∫_{-∞}^{∞} \left|Ae^{-a\left|x\right|}\right|^2 \ \mathrm{d}x &= 1\\
    \left.\frac{2A^2}{-2a}e^{-2ax}\right\rvert_{0}^{∞} &= 1\\
    \frac{A^2}{a} &= 1\\
    A = \sqrt{a}  
\end{align*}
Vi får den normaliserte bølgefunksjonen $Ψ_n$
\[
\underline{\underline{Ψ_n(x,0) = \sqrt{a} e^{-a\left|x\right|}}}
\]
\subsubsection*{b)} 
Vi bruket likning 2.104 fra Griffiths for $ϕ(k)$
\begin{align*}
ϕ(k) &= \frac{1}{\sqrt{2π}} ∫_{-∞}^{∞} Ψ(x,0)e^{-ikx} \ \mathrm{d}x
\\
ϕ(k) &= \sqrt{\frac{a}{2π}} ∫_{-∞}^{∞} e^{-a\left|x\right|} e^{- ikx}  \ \mathrm{d}x
\\
&\text{Vi skriver om siste faktoren på sinus og cosinus form}\\
ϕ(k) &= \sqrt{\frac{a}{2π}} ∫_{-∞}^{∞} e^{-a\left|x\right|} \left(cos(kx) - i sin(kx)\right) \ \mathrm{d}x
\\
&\text{Videre endrer vi grensene på integrale
}\\
ϕ(k) &= 2\sqrt{\frac{a}{2π}} ∫_{0}^{∞} e^{-ax}\left(\cos(kx) - i \sin(kx)\right) \ \mathrm{d}x
\\
&\text{Ettersom sinus funksjonen er en odde funksjon vil denne delen av integralet forsvinne. Cosinus leddet skriver vi om til eksponensial form}\\
ϕ(k) &= \sqrt{\frac{a}{2π}} ∫_{0}^{∞} e^{-ax} \left(e^{ikx} + e^{-ikx}\right) \ \mathrm{d}x
\\
ϕ(k) &= \sqrt{\frac{a}{2π}} ∫_{0}^{∞} \left(e^{x(ik - a)} + e^{-x(ik + a)}\right) \ \mathrm{d}x
\\
ϕ(k) &= \sqrt{\frac{a}{2π}} \left(\left.\frac{e^{x(ik-a)}}{ik - a} - \frac{e^{-x(ik+a)}}{ik + a}\right)\right\rvert_{0}^{∞}
\\
&\text{Evaluerer integralet og finner felles nevner}\\
ϕ(k) &= \sqrt{\frac{a}{2π}} \left(-\frac{1}{ik-a} + \frac{1}{ik+a}\right) = \sqrt{\frac{a}{2π}} \frac{-(ik+a) + (ik-a)}{(ik)^2 - a^2}
\\
ϕ(k) &= \sqrt{\frac{a}{2π}} \frac{-2a}{-k^2 - a^2}\\
ϕ(k) &= \underline{\underline{\sqrt{\frac{a}{2π}} \frac{2a}{k^2 + a^2}}}
\end{align*}


\subsubsection*{c)}
Vi bruker likning 2.101 fra Griffiths for $Ψ(x,t)$ og setter inn vårt utrykk for $ϕ(k)$
\[
Ψ(x,t) = \frac{1}{\sqrt{2π}} ∫_{-∞}^{∞} ϕ(k) e^{i \left(kx - \frac{ℏk^2}{2m}t\right)} \ \mathrm{d}k = \frac{1}{\sqrt{2π}} 2a\sqrt{\frac{a}{2π}} ∫_{-∞}^{∞} \frac{1}{k^2 + a^2} e^{i\left(kx - \frac{ℏk^2}{2m}t\right)} \ \mathrm{d}k
\]
\[
Ψ(x,t) = \frac{\sqrt{a^{3}}}{π} ∫_{-∞}^{∞} \frac{1}{k^2+a^2} e^{i\left(kx - \frac{ℏk^2}{2m}t\right)} \ \mathrm{d}k
\]

\subsubsection*{d)}
I grensene hvor $a$ er veldig liten, vil vi ha en større usikkerhet i hvor partikkelen befinner seg, men vi har god kontroll på dets bevegelsesmengde. Dette er fordi $Ψ$ blir veldig bre, og $ϕ$ blir smal. I grensen hvor $a$ er veldig stor vil vi ha en større usikkerhet i bevegelsesmengden til partikkelen, men vi har god kontroll på hvor den befinner seg. Dette er fordi $Ψ$ blir veldig smal, og $ϕ$ blir bred.

\subsection*{Oppgave 4}
\paragraph*{$\mathbf{L_1}:$}
\[
- \frac{ℏ^2}{2m} \frac{\mathrm{d}^2}{\mathrm{d}x^2} ψ_1 = E ψ_1 \quad \Bigg| ⋅ ψ_2
\]
\[
- ψ_2\frac{ℏ^2}{2m} \frac{\mathrm{d}^2}{\mathrm{d}x^2} ψ_1 = ψ_2E ψ_1
\]
\paragraph*{$\mathbf{L_2}:$}
\[
- \frac{ℏ^2}{2m} \frac{\mathrm{d}^2}{\mathrm{d}x^2} ψ_2 = E ψ_2 \quad \Bigg| ⋅ ψ_1
\]
\[
- ψ_1\frac{ℏ^2}{2m} \frac{\mathrm{d}^2}{\mathrm{d}x^2} ψ_2 = ψ_1E ψ_2
\]
\newline
\[
L_2 - L_1 = - ψ_1\frac{ℏ^2}{2m} \frac{\mathrm{d}^2}{\mathrm{d}x^2} ψ_2 + ψ_2\frac{ℏ^2}{2m} \frac{\mathrm{d}^2}{\mathrm{d}x^2} ψ_1 = \underbrace{ψ_1E ψ_2 - ψ_2E ψ_1}_{0}
\]
\[
- ψ_1\frac{ℏ^2}{2m} \frac{\mathrm{d}^2}{\mathrm{d}x^2} ψ_2   + ψ_2\frac{ℏ^2}{2m} \frac{\mathrm{d}^2}{\mathrm{d}x^2} ψ_1 = 0
\]
\[
ψ_2 \frac{\mathrm{d}^2}{\mathrm{d}x^2} ψ_1 - ψ_1 \frac{\mathrm{d}^2}{\mathrm{d}x^2} ψ_2 = 0
\]
Ettersom $ψ_2 ≠ ψ_1 ≠ 0$ må $\displaystyle ψ_2 \frac{\mathrm{d}}{\mathrm{d}x} ψ_1 $ og $\displaystyle ψ_1 \frac{\mathrm{d}}{\mathrm{d}x} ψ_2 $ være konstant, slik at deres dobbelt derivert blir 0. 
Videre vet vi at $ψ → 0$ ved $\pm∞$ for normerbare løsninger. Da vet vi at den derivert ikke er hvilken som helst konstant, men må være 0. Da kan vi skrive $ψ_2 = kψ_1$. Det betyr at løsningene ikke er distinkte. 

\subsection*{Oppgave 5}
Vi starter med Schrödingerligningen
\[
\frac{\mathrm{d}^2}{\mathrm{d}x^2} ψ(x) = -k^2 ψ(x) \quad , \quad k = \frac{\sqrt{2mE}}{ℏ^2}
\]

Videre skriver vi $ψ$ på eksponensial form. 
\[
ψ(x) = Ae^{ikx} + Be^{-ikx}
\]
Vi har at $ψ(x) = ψ(x + L)$
\[
Ae^{ikx} + Be^{-ikx} = Ae^{ik(x+L)} + Be^{-ik(x+L)} = Ae^{ikx}e^{ikL} + Be^{-ikx}e^{-ikL}
\]
Dette gjelder for alle x. Videre setter vi $x=0$ for enkelheten skyld.
\[
A + B = Ae^{ikL} + Be^{-ikL}
\]
Ettersom vi kunne ha skrevet $ψ$ som en cosinus funksjon vet vi også at ligningen over stemmer for $x = \frac{π}{2k}$. 
\[
Ae^{iπ / 2} + B^{- iπ / 2} = Ae^{ikL} + Be^{-ikL} = 0
\]
\[
iAe^{ikL} - Be^{-ikL} = iA - iB
\]
\[
Ae^{ikL} - Be^{-ikL} = A - B
\]
\[
2Ae^{ikL} = 2A
\]
Dette stemmer bare hvis $A=0$, eller $e^{ikL}=1$ for $kL = 2nπ$ der $n =  0,±1,±2,±3, \ldots $. Hvis $A=0$ må $Be^{-ikL} = B$ som gir samme resultat. Det betyr at for hver $n$ er det to løsninger $ψ_n^{+} = Ae^{i(2nπx /L)}$ og $ψ_n^{-} = Be^{-i(2nπx /L)}$, ved unntak av $n=0$ der det bare er én løsning. Ved normalisering av $\left|ψ_{\pm}\right|^2$ i intervallet $[0,L]$ får vi $A=B = 1 /\sqrt{L}$ hvor alle andre løsninger med samme energi er en lineær kombinasjon av disse. Da få rvi et utrykk for $ψ$ og energi $E$
\[
ψ_{±} \frac{1}{\sqrt{L}} e^{±i(3nπx / L)} \quad , \quad E = \frac{ℏ^2k^2}{2m} = \frac{ℏ^2(2nπ / L)^2}{2m}
\]

Teoremet krever at $ψ → 0$ ved $\pm∞$. Dette stemmer ikke i vårt tilfelle og vi kan derfor få to bundne tilstander med samme energi. 

\end{document}