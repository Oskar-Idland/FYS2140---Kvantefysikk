\documentclass[norsk]{article}
\usepackage{amsmath}
\usepackage[mathletters]{ucs}
\usepackage[utf8x]{inputenc}
\usepackage[margin=1.5in]{geometry}
\usepackage{enumerate}
\newtheorem{theorem}{Theorem}
\usepackage[dvipsnames]{xcolor}
\usepackage{pgfplots}
\setlength{\parindent}{0cm}
\usepackage{graphics}
\usepackage{graphicx} % Required for including images
\usepackage{subcaption}
\usepackage{bigintcalc}
\usepackage{pythonhighlight} %for pythonkode \begin{python}   \end{python}
\usepackage{appendix}
\usepackage{arydshln}
\usepackage{physics}
\usepackage{tikz-cd}
\usepackage{booktabs} 
\usepackage{adjustbox}
\usepackage{mdframed}
\usepackage{relsize}
\usepackage{physics}
\usepackage[thinc]{esdiff}
\usepackage{fixltx2e}
\usepackage{esint}  %for lukket-linje-integral
\usepackage{xfrac} %for sfrac
\usepackage{hyperref} %for linker, må ha med hypersetup
\hypersetup{colorlinks=true}
\usepackage[noabbrev, nameinlink]{cleveref} % to be loaded after hyperref
\usepackage{amssymb} %\mathbb{R} for reelle tall, \mathcal{B} for "matte"-font
\usepackage{listings} %for kode/lstlisting
\usepackage{verbatim}
\usepackage{graphicx,wrapfig,lipsum,caption} %for wrapping av bilder
\usepackage{mathtools} %for \abs{x}
\usepackage[norsk]{babel}
\definecolor{codegreen}{rgb}{0,0.6,0}
\definecolor{codegray}{rgb}{0.5,0.5,0.5}
\definecolor{codepurple}{rgb}{0.58,0,0.82}
\definecolor{backcolour}{rgb}{0.95,0.95,0.92}
\lstdefinestyle{mystyle}{
    backgroundcolor=\color{backcolour},   
    commentstyle=\color{codegreen},
    keywordstyle=\color{magenta},
    numberstyle=\tiny\color{codegray},
    stringstyle=\color{codepurple},
    basicstyle=\ttfamily\footnotesize,
    breakatwhitespace=false,         
    breaklines=true,                 
    captionpos=b,                    
    keepspaces=true,                 
    numbers=left,                    
    numbersep=5pt,                  
    showspaces=false,                
    showstringspaces=false,
    showtabs=false,                  
    tabsize=2
}

\lstset{style=mystyle}
\author{Oskar Idland}
\title{Oblig 9}
\date{}
\begin{document}
\maketitle
\newpage
\section*{\underline{A Diskusjonsoppgaver}}
\subsection*{Oppgave 1}
\subsubsection*{a)}
Vi kan måle elektronets posisjon $x$ med operatoren $\hat{x}$, bevegelsesmengde $p$ med operatoren $\hat{p}$ og energien $E$ ved operatoren $\hat{H}$. 

\subsubsection*{b)}
observabelen $x$ og $E$ har operatorer med egenverdier på følgende form. 
\[
\hat{x}ψ = xψ \quad , \quad \hat{H}ψ = Eψ
\]
Dette gjelder ikke bevegelsesmengde som et uttrykt på følgende måte.  
\[
\hat{p}ψ = -iℏ\frac{∂}{∂x}ψ
\]

\subsubsection*{c)}
Elektronet har bare skarpe verdien for observabelen $E$ og $p$ ettersom elektronet bare kan ha diskrete distanser fra kjernen som gir diskret bevegelsesmengde og energi. 

\subsubsection*{d)}
$n$ beskriver energinivået til elektronet og dets distanse fra kjernen. $n$ er alltid et heltall større eller lik 1, ettersom det beskriver elektronskall. Jo høyere $n$ jo lenger unna kjernen.
\newline\newline
$l$ beskriver størrelsen på elektronets orbitale angulær moment gitt ved $L = ℏ\sqrt{l(l+1)}$. $l$ er alltid et naturlig tall i intervallet $[0, n-1]$. Tallet forteller oss hvilket underskall elektronet befinner seg i og dets form. Det enkleste eksempelet er når $n=1$. Da kan $l$ kun være 0. Det innerste skallet har bare et underskall ($1s$). For $n=2$ kan $l$ være 0 eller 1. Dette er  det nest innerste skalle som har to underskall $s_2$ og $p_2$.
\newline\newline
$m$ er det magnetiske kvantetallet og beskriver antall orbitaler i et underskall, dets orientering samt det orbitale angulær momentet om $z$-aksen gitt ved $L_z = mℏ$. $m$ er alltid et heltall i intervallet $[-l,l]$. Hvis $n=2$ er $l$ enten 0 eller 1. Da kan $m$ være enten -1, 0 eller 1. Dette gir 3 mulige orbitaler i andre energinivå. $p_x, p_y\ \&\ p_0$. $m$ påvirker ikke elektronet sin energi med mindre det er i et magnetfelt. 

\subsubsection*{e)}
Vi vet at $E_n = -E_0 / n^2$. Med $\displaystyle E = - \frac{m_ee^2k^2_{e}}{18ℏ^2}$ må $n = 3$. 
Da kan vi finne $l ∈ [0,1,2]$. Vi løser følgende. 
\[
L^2 = ℏ^2\left(\sqrt{l(l+1)}\right)^2 = 6ℏ^2
\]
\[
l^2 + l = 6 \quad , \quad l = 2 \ /\ l=-3
\]
Bare $l=2 ∈ [0,1,2]$. Da gjenstår bare $m ∈ [-2,2]$. Vi løser følgende. 
\[
L_z = mℏ = 3ℏ \quad , \quad m = 3
\]
Den eneste løsningen for $m \not∈ [-2,2]$. Måleresultatene er dermed \underline{ikke gyldig}. 

\section*{\underline{B Regneoppgaver}}
\subsection*{Oppgave 2}
\subsubsection*{a)}
Et sentralsymmetrisk potensial $V(r)$ er et potensial som bare er avhengig av avstanden $r$ til kilden til potensialet plassert i origo. Et eksempel på dette er gravitasjonspotensialet, eller den potensielle energien et elektron får fra en positiv partikkel. 
$Y_{l}^{m}(ϕ,θ)$ er den samme enten vi ser på et fritt eller bundet elektron ettersom funksjonen bare er avhengig av vinklene $ϕ$ og $θ$. Et bundet elektron er bare bundet i avstanden $r$ fra kjernen.

\subsubsection*{b)}
\[
- \frac{ℏ^2}{2m_e} \frac{d^2}{dr^2}rR(r) + \left[- \frac{k_ee^2}{r} + \frac{ℏ^2(l(l+1))}{2 m_e r^2}\right]rR(r)  = Er(R(r))
\]
Vi setter inn den radielle funksjon $R(r) = re ^{-γr}$ i venstre side av likningen. 

\begin{equation}\label{eq: b_1}
    - \frac{ℏ^2}{2m_e} \frac{d^2}{dr^2}r^2e^{-γr} + \left[- \frac{k_ee^2}{r} + \frac{ℏ^2(l(l+1))}{2 m_e r^2}\right]r^2e^{-γr}  = Er^2e^{-γr}
\end{equation}

Vi begynner med å derivere $\displaystyle \frac{\mathrm{d}^2}{\mathrm{d}r^2}r^2e^{-γr}$
\[
\frac{\mathrm{d}^2}{\mathrm{d}r^2}r^2e^{-γr} = \frac{\mathrm{d}}{\mathrm{d}r}\left(2re^{-γr} -γr^2e^{-γr}\right) = e^{-γr}\left(γ^2r^2 - 4γr + 2\right)
\]
Vi setter dette tilbake i \cref{eq: b_1}.
\[
- \frac{ℏ^2}{2 m_e} e^{-γr}\left(γ^2r^2 - 4γr + 2\right) + \left[- \frac{k_ee^2}{r} + \frac{ℏ^2(l(l+1))}{2 m_e r^2}\right]r^2e^{-γr}
\]

\[
- \frac{ℏ^2}{2 m_e}\left(γ^2r^2 - 4γr + 2\right) + \left[- \frac{k_ee^2}{r} + \frac{ℏ^2(l(l+1))}{2 m_e r^2}\right]r^2 = E_nr^2
\]
\[
- ℏ^2\left(γ^2r^2 - 4γr + 2\right)  -  2 m_ek_ee^2r + ℏ^2(l(l+1)) - 2 m_eE_nr^2 = 0 
\]
\[
-ℏ^2(-4γr+2) - 2 m_e k_ee^2r + ℏ^2(l(l+1)) - \left(2 m_e E_n - ℏ^2γ^2\right)r^2 = 0
\]
\[
-2ℏ^2  + (4γℏ^2 - 2 m_e k_ee^2)r + ℏ^2(l(l+1)) - \left(2 m_e E_n - ℏ^2γ^2\right)r^2 = 0
\]
\[
\underbrace{ℏ^2(l(l+1) - 2)}_{\text{Ledd 1}} + \underbrace{(4γℏ^2 - 2 m_e k_ee^2)r}_{\text{Ledd 2}} - \underbrace{\left(2 m_e E_n - ℏ^2γ^2\right)r^2}_{\text{Ledd 3}} = 0
\]


Hvis likningen skal gjelde for alle $r$ må leddene være lik 0. Vi begynner med ledd 2. 
\[
4γℏ^2 - 2 m_e k_ee^2 = 0
\]
\[
γ = \frac{m_e k_ee^2}{2ℏ^2}
\]
Vi fortsetter med ledd 3. 
\[
2 m_e E_n - ℏ^2γ^2 = 0
\]
Vi setter inn $E_n$ og $γ$. 
\[
2 m_e \frac{E_0}{n^2} - ℏ^2\left(\frac{m_e k_ee^2}{2ℏ^2}\right)^2 = 2 \frac{E_0}{n^2} - \frac{m_e k_e^2e^4}{4ℏ^2} = 0
\]
\[
2 m_e \frac{m_ek_e^2e^4 }{2ℏ^2} \frac{1}{n^2} - \frac{m_e k_e^2e^4}{4ℏ^2} = 0
\]
\[
\frac{1}{n^2} = \frac{1}{4}
\]
\[
n = 2
\]
Vi avslutter med ledd 1.
\[
l(l+1) - 2 = 0
\] 
\[
l_1 = -2 \text{ eller } l_2 = 1
\]
Ettersom bare $l_2 ∈ [0,2]$ er det den eneste løsningen. Da vet vi at $R(r) = re^{-γr}$ er en gyldig løsning for $n = 2$, $l = 1$ og $\displaystyle γ = \frac{m_e k_ee^2}{2ℏ^2}$. 

\subsubsection*{c)}
Vi er vant med å regne i kartesiske koordinater i én dimensjon. Når vi bruker sfæriske koordinater i tre dimensjoner ser utrykket for sannsynlighetsfordelingen litt annerledes ut. Sannsynligheten for å finne elektronet i et volumelement $dV$ er gitt ved $\left|Ψ\right|^2 dV$. I sfæriske koordinater er $dV = r^2 \sin θ\ dr\ dθ\ dϕ$. Siden vi kan separere løsningen av Schrodinger likningen til en radiell del kan vi se bort ifra sinus utrykket i den radielle delen og får da en radiell sannsynlighet gitt ved. 
\[
P(r) =  \left|Ψ\right|^2 r^2 = r^2 R^2(r)
\]
Videre finner vi den mest sannsynlige radiusen. 
\[
P(r)_{\text{max}} := \frac{\mathrm{d}}{\mathrm{d}r} r^2 R^2(r) = 0
\]
\[
\frac{\mathrm{d}}{\mathrm{d}r} r^4 e^{-2γr} = 0
\]
\[
4r^3e^{-2γr} - 2γr^4e^{-2γr} = 0
\]
\[
2r^3e^{-2γr}\left(2 - γr\right) = 0
\]
Vi ser vi har en løsing for $r_1 = 2 / γ$ og for $r_2 = 0$. Vi vet at $r=0$ ikke er en løsning ettersom partikkelen ikke kan okkupere sentrum av atomet. 
\[
r = \frac{2}{γ} = \frac{4ℏ^2}{m_e k_ee^2} = 4a_0
\]


\end{document}