\section{10 Forelesnings Notater}
\paragraph{\underline{Sentrale kunnkaper}}
\begin{enumerate}
    \item Sette opp og utlede løsning til TUSL for uendelig dyp brønn, beregne egentilstandene og forstå ortogonalitet og kronecker delta.
    \item Lage superposisjon av egentilstander
    \item Forstå figurer der vi kombinerer potensial, energi og bølgefunksjon, beregne $\left<h\right>, σ_{H}$ og sannsynligheten for å måle n-te tilstand
    \item Trig funksjoner, deres integrerte, deriverte og sum
\end{enumerate}

\subsection{Potensialet til uendelig dyp brønn}
\[
\hat{Q} → \hat{H} = \left(- \frac{ℏ^2}{2m}\frac{\mathrm{d}^2}{\mathrm{d}x^2} + V(x)\right) \quad , \quad Ψ_{E} ≡ Ψ_{n}(x,t) = ψ_n(x)f_n(t)
\]
Vi separerer bølgefunksjonen i to deler, en som er avhengig av posisjon og en som er avhengig av tid. 
\[
\hat{H}ψ_n(x) = E_nψ_n(x) \quad , \quad φ_n(t) = E^{-iE_{n} t / ℏ}
\]
\paragraph{Generell bølgefunksjon}
\[
Ψ(x,t) = \sum_{n=1}^{\infty}c_nψ_n(x)e^{-iE_nt/ℏ}
\]
Ettersom vi har separert tid og rom kan vi sette $t = 0$ for å gjøre om eksponenten til 1. Da blir bølgefunksjonen
\[
Ψ(x,0) = \sum_{n=1}^{}c_nψ_n(x) \quad , \quad ∑_{n}^{} \left|c_n\right|^2 = 1 = ∫ ψ^{*}_n(x)ψ_n(x,0)\ dx
\]
$\left|c_n\right|^2 $ er sannsynligheten for å måle n-te tilstand. Bølgefunksjonen er for abstrakt til å finne energien, men vi kan finne gjennomsnittsenergien $\left<H\right>$. 
\[
\left<H\right> = ∫ Ψ^{*}(x,t)\hat{H}Ψ(x,t)\ dx = ∑_{n}^{} E_n \left|c_n\right|^2
\]
Hvis alle tilstandene har samme energi kan dette faktoriserer ut og vi får. 
\[
Ψ(x,t) = \left(∑_{}^{} c_n ψ_n(x)\right)e^{-iE_nt/ℏ}
\]
\[
\left<p\right> = ∫ Ψ^{*}(x,t)\hat{p}Ψ(x,t)\ dx ≠  ∑_{n}^{} p_n \left|c_n\right|^2
\]
Det går ikke alltid å summere som vanlig når vi skal finne forventningsverdien til bevegelsesmengden $p$. 

\paragraph{Triks for å spare tid}
Annta et potensialet som variere, men er konstant i noen intervaller. Da vil TUSL gitt ved 
\[
\left(- \frac{ℏ^2}{2m}\frac{\mathrm{d}^2}{\mathrm{d}x^2} + V_0\right)ψ_n(x) = E_nψ_n(x)
\]
\[
- \frac{ℏ^2}{2m}\frac{\mathrm{d}^2}{\mathrm{d}x^2} ψ_n = \underbrace{\left(E_n - V_0\right)}_{K_n} ψ_n(x)
\]
\[
ψ_n(x) = A e^{i k_nx} + B e^{-i k_nx}
\]
\[
K_n = E - V_0 = \frac{ℏ^2 k_n^2}{2m} ⇒ k_n = \sqrt{\frac{2m}{ℏ^2}(E_n - V_0)}
\]
Dette er mulig ettersom TULS er lokal med hensyn på $x$. Konstantene A og B er globale for hele funksjonen. 
Vi har noen forskjellige cenario. 
\begin{enumerate}
    \item $E > V_0$\\
    Dette er det vanligste tilfellet. Da vil $k_n$ være reelt og $ψ_n(x)$ er planbølgen. 
    \item $E < V_0$\\
    her er $k_n$ imaginært $κ_n = i k_n = \displaystyle \sqrt{- \frac{2m(E- V_0)}{ℏ^2}}$ som er reelt.\\
    $ψ_n(x) = \underbrace{A e^{κ_n x} + Be^{-κ_n x}}_{\text{Reelle eksp. funk.}}$.\\ 
    Det er bare i et lite område hvor energien $E$ er mindre enn potensiell energi $V$ 
    \item $E = V_0$
    $ψ_n(x) = A + Bx$
\end{enumerate}

\paragraph{Cases}
\[
V(x) = 0 \quad , \quad 0 \le x \le a \quad , \quad V(x) = ∞
\]
Vi skal finne en bølgefunksjon som er kontinuerlig både i og utenfor brønn. I brønnen er funksjonen 
\[
ψ_n(x) = A e^{i k_nx} + B e^{-i k_nx}
\]
Utenfor er den 0. Vi løser deretter 
\[
ψ(0) = 0 ⇒ B = -A
\]
Så løser vi for grensene 
\[
ψ(a) = A e^{i k_na} - A e^{-i k_na} = A \left(e^{i k_na} - e^{-i k_na}\right)
\]
Dette kjenner vi igjen som en sinus funksjon. 
\[
2iA\sin (k_na) = 0 ⇒ k_n = \frac{nπ}{a} \quad , \quad n = 1, 2, 3 \ldots , ∞
\]
\[
E_n = \frac{ℏ^2 k_n^2}{2m} = \frac{h^2}{2m} \frac{π^2}{a^2}n^2
\]
Vi normaliserer sinus funksjonen på følgende måte. 
\[
C \sin (k_na) = 0
\]
\[
1 = ∫ ψ_n^{*}ψ_n \ \mathrm{d}x = ∫_{0}^{a} C \sin (k_nx) ⋅ C \sin (k_nx) \ \mathrm{d}x = C^2 ∫_{0}^{a} \sin ^2 (k_n x) \ \mathrm{d}x
\]
\[
1 = C^2 \left.\left(\frac{x}{2} - \frac{1}{4 k_n} \sin (2 k_nx)\right)\right\rvert_{0}^{a} = C^2 \left(\frac{a}{2}\right) ⇒ C = \sqrt{\frac{2}{a}}
\]
Dette gir bølgefunksjonen
\[
ψ_n(x) = \sqrt{\frac{2}{a}} \sin \left(\frac{nπ}{a}x\right) \quad , \quad n = 1, 2, 3 \ldots 
\]
Egenfunksjonene er ortogonale og produktet vil alltid være 0, hvis ikke de har samme tilstand. 
\[
δ_{nm} = \begin{cases}
  1, &\text{ if }n = m\\
  0, &\text{ if }n ≠  m
\end{cases}
\]
En grei regel er at
\[
E_1 = \frac{ℏ^2 k^2}{2m} \frac{π^2}{a^2} \ \quad , \quad E_2 = E_1 ⋅ 2^2 \quad , \quad E_3 = E_1 ⋅ 3^2 \quad , \quad E_4 = E_1 ⋅ 4^2 \quad , \quad \ldots
\]
Da kan vi finne bølgefunksjonen
\[
Ψ(x,0) = ∑_{n}^{} c_n ψ_n(x) = ∑_{n}^{} c_n \sqrt{\frac{2}{a}} \sin \left(\frac{nπ}{a}x\right)
\]
\paragraph{Typiske eksamensoppgaver}
\begin{enumerate}
    \item å lage $Ψ(x,0)$ fra egenfunksjoner. 
    \item Anta $Ψ(x,0)$ hva er sannsynligheten at vi får n-te tilstand ved måling? Da må vi beregne $c_n$. ved $\displaystyle c_n = ∫ ψ_n^{*}(x) Ψ(x,0) \ \mathrm{d}x \frac{2}{nπ} \left(1 - \cos \left(\frac{nπ}{2}\right)\right)$
\end{enumerate}

\[
Ψ(x,0) = 
\begin{cases}
    \sqrt{\frac{2}{a}} &\text{ når } 0 \le x \le a / 2 \\
    0              & \text{ ellers } 
\end{cases}
\]

De første tre tilstandene er mye mer sannsynlig enn resterende og er ofte nok for en god approksimasjon. 


\subsection*{Dirac-delta-funksjonen}
\[
δ(x-a) = 
\begin{cases}
  ∞, &\text{ if }x = a\\
  0, &\text{ if }x ≠  a 
\end{cases}
\]
\[
∫_{-∞}^{∞} δ(x-a) \ \mathrm{d}x = 1
\]
\[
∫_{-∞}^{∞} f(x) ϕ(x-a) \ \mathrm{d}x = f(a)
\]
\subsection*{Ortogonale Egenfunksjoner}
\[
∫ ψ_n^{*}(x) ψ_m(x) \ \mathrm{d}x = δ_{nm}
\]
\[
Ψ(x,0) = ∑_{n}^{} c_n ψ_n(x)
\]
\[
ψ_m = ∑_{n}^{} c_n ψ_n \quad , \quad c_n = \begin{cases}
  1, &\text{ når }m = n\\
  0, &\text{ ellers }
\end{cases}
\]
\[
1 = ∫ ψ_m^{*}(x) ψ_m(x) \ \mathrm{d}x \overset{\underset{\text{antar}}{}}{=} ∫ ψ_m^{*} ∑_{n ≠ m}^{} c_n ψ_n \ \mathrm{d}x = ∑_{n ≠ m}^{} c_n \underbrace{∫ ψ_m^{*}ψ_n \ \mathrm{d}x}_{δ_{mn}} = 0
\]
