\section{08 Forelesnings Notater}
\paragraph{\underline{Sentrale kunnskaper}}
\begin{enumerate}
    \item Forholdene i tabell 1.4 i kompendiet 
    \item Beregne forventningsverdi og standardavvik for diskrete og kontinuerlige variabler
    
    \item Forstå superposisjon som for frie partikler er oppbygget av planbølger
    \item Hva som skjer når man måler en superposisjon av tilstander 
    \item Københavnstolkningen 
\end{enumerate}

Den vanlige maxwell likningen fungerer ikke når vi setter inn partikkelegenskapene til fotonet. Vi får et restledd på $\displaystyle \left( mc^2 \right) ^2$ som gjør at den fulle likningen ikke lenger er null. Vi ser at hvis vi legger til en faktor av $\displaystyle - \left( \frac{i}{ℏ} \right) \left( mc^2 \right) ^2$ ved å ta hensyn til hvileenergien før vi deriverer og likningen løser seg. 

\subsection{Tidsavhengige bølgeligninger}
\paragraph{Maxwells ligning for klassiske elektromagnetiske bølger (relativistisk)}
\[
\left( \frac{∂^2 }{∂ t^2} - c^2 \frac{∂^2 }{∂ x^2} \right) ϵ(x,t) = 0
\]
\paragraph{Klein-Fock-Godronds relativistiske lignign for frie partikler med masse (spinn løse)}
\[
    \left(\frac{∂^2 }{∂ t^2} - c^2 \frac{∂^2 }{∂ x^2} + \left(\frac{mc^2}{ℏ}\right)^2\right) Ψ(x,t) = 0
\]

\subsection{Bølgepakke for fri partikkel}
\[
\left(-i ℏ \frac{∂ }{∂ t} - \frac{ℏ^2}{2m} \frac{∂^2 }{∂ x^2} \right) Ψ(x,t) = 0
\]
\subsection{Generell løsning av bølgepakke av planbølger}
\[
Ψ(x,t) = \sum_{n}^{} c_n e^{i \left( k_n x - ω_n t  \right) }
\]
Kan også skrives 
\[
Ψ = \sum_{n}^{} c_n e^{i \left( p_n x - E_n t \right)/ℏ}
\]
\[
ψ = \sum_{n}^{} c_n e^{i \left( k_n x - ω_n t \right)}
\]
Her er $\displaystyle k_n = \frac{2π}{λ} = \frac{p}{ℏ}\quad , \quad ω = \frac{2πc}{λ} = \frac{E}{ℏ}$. 
\paragraph{Superposisjonering av egentilstander}
En bølgefunksjon er en superposisjon av egentilstander. Hvis vi måler bølgen kollapser den til en enkelt bølge med bestemt energi, bevegelsesmengde og bølgelengde. Hvordan dette skjer vet vi ikke. Konstanten $\left|c_n\right|^2 $ er sannsynligheten for kollaps til planbølgen med $λ_n = 2π / k_n$. Da følger det naturlig at $\left|c_n\right|^2 = 1$

\subsection{Tolkninger}
\paragraph{Realist}
Partikkelen hadde en forutsagt bane før måling og det er en skjult variabel som styrer dette. (Har blitt motbevist)
\paragraph{Ortodoks / Københavnstolkningen}
Bølgefunksjonen er ikke ekte. Målingen fikk den til  å kollapse. Målingen påvirker partikkelen, uten den også 
\paragraph{Agnostisk}
Har ikke noe å si ettersom vi må måle uansett
\newline \newline
Den mest aksepterte tolkningen er Københavnstolkningen. 
