\section{06 Forelesnings Notater}
\paragraph{\underline{Sentrale Kunnskaper}}
\begin{itemize}
    \item Beskrive de Broglies hypotese og forstå hvordan de Broglies teori bidrar til Bohr's atommodell
    \item Beskrive resultatene fra dobbelspalte-eksperimentet 
    \item Beskrive hvordan eksperiment viser på partikkel- og bølge-egenskaper til materien
    \item Forstå sannsynligheten til treff, og interferensmønster på skjermen bak. 
    \item Beskrive interferensfenomenet i Davisson-Germer eksperimentet og Thomson-diffraksjon. 
\end{itemize}

\subsubsection{de Broglies hyptotese og atom}
\paragraph{Hypotese}
Hvis lys/fotoner oppfører seg som både bølge og partikkel må det samme gjelde materie. 

Da må materie ha en energi $E$ og en bevegelsesmengde $ρ$. Vi ser på dette i kontekst av elektroner 
\[
E = hν = ℏω, \quad ω = 2πν \text{ (vinkelfrekvens)}
\]
\[
ρ = \frac{h}{λ} = ℏk, \quad k = \frac{2π}{λ} \text{ (bølgetall)} 
\]
\[
nλ  = \underbrace{2πr}_{\text{omkrets}} 
\]
Det betyr at omkretsen til sirkelen er en multippel av bølgelengden $λ$. 

Vi slår sammen bølgeversjonen av bevegelsesmengde og partikkel versjonen av bevegelsesmengde.
\paragraph{Bølge} 
\[
ρ = \frac{h}{λ} = ℏk = \frac{h}{2πr}n = ℏn = mv ⇒ ℏn = \underbrace{rmv}_{\text{Ang. momentum}} = L = ℏn
\]
Da er angulært moment også kvantisert. 

Vi vet fortsatt ikke nøyaktig \textit{hva} en partikkel er, men bare dets tilstand. 

\paragraph{Oppsumering}
\begin{itemize}
    \item Elektronet tilordnes både partikkel- og bølgeegenskaper. 
    \item Forklarer Bohr's kvantisering av angulærmomentum.
    \item Bølgeegenskapen er en fundamental forskjell mot Bohr's partikkel-lignende elektron. 
    \item Kvantisering av bølgelengden er det som gjør kvantisering av energien mulig. 
\end{itemize}


\subsubsection{Dobbeltspalte-eksperimentet}
Fenomenet kan bare forklares, ikke begrunnes. Kort oppsummert vil vi se et interferens mønster uavhengig av om vi sender ut bølger, eller kvantiserte energi pakker (fotoner) av elektromagnetisk stråling. Det har ikke noe å si hvor stort mellomrom mellom utskytningen av fotoner. Vi får uansett det samme mønsteret. Hvor hver partikkelen treffer på skjermen bak, er uavhengig av hverandre og helt tilfeldig. Hvorfor dette skjer? \underline{Vi vet ikke}. Dette kan gjennomføres med vannbølger, lydbølger, elektromagnetisk stråling, eller atomer. Hvis vi setter en sensor foran bare én av spaltene vil partiklene passere ca. halvparten av gangene, men interferens mønstrete opphører. 

\begin{itemize}
    \item Enkelt-partikler må dermed \textit{interferere} med seg selv. Som om det passerer gjennom begge spalter samtidig. 
    \item Hver partikkel har en eget innebygd sannsynlighet. 
    \item Vi vet ikke hvilken vei det passerer gjennom spaltene, eller om det er begge samtidig. 
    \item Etter treff, vet vi heller ikke hvilken vei det tok.
    \item Målingen av banen til partikkelen ⇒ interferens opphører. 
    \item Måling endrer sannsynligheten. 
    \item Partikkelen vet ikke selv hvilken vei det skal ta. 
\end{itemize}

\paragraph{Hidden variables - teorien}
Partikkelen må selv vite hvilken spalte den skal gjennom, og den er da underkastet en deterministisk oppførsel. Vi kjenner bare ikke den underliggende fysikken. \\\\
"\textit{Gud spiller ikke terning}" \\- Albert Einstein. \\\\
"\textit{"Slutt å fortelle Gud hva han skal gjøre}" \\- Niels Bohr. 

\subsubsection{Sanssynlighetstettheten}
\[
\frac{\mathrm{d}^2 Ψ(x)}{\mathrm{d}x^2} = E ⋅ Ψ (x)
\]

